
\documentclass[reqno,11pt]{amsart}

%\usepackage{color,graphicx}
%\usepackage{mathrsfs,amsbsy}
\usepackage{amssymb}
\usepackage{amsmath}
\usepackage{amsfonts}
\usepackage{bm}
\usepackage{graphicx}
\usepackage{amsthm}
\usepackage{enumerate}
\usepackage[mathscr]{eucal}
\usepackage{float}
\usepackage{mathrsfs}
\usepackage{multicol}
\usepackage[all,pdf]{xy}
\usepackage[a4paper,left=3cm,right=3cm]{geometry}
\usepackage{xcolor}
%\usepackage[notcite,notref]{showkeys}

% showkeys  make label explicit on the paper

\makeatletter
\@namedef{subjclassname@2010}{%
  \textup{2010} Mathematics Subject Classification}
\makeatother

\numberwithin{equation}{section}

\theoremstyle{plain}
\newtheorem{theorem}{Theorem}[section]
\newtheorem{lemma}[theorem]{Lemma}
\newtheorem{proposition}[theorem]{Proposition}
\newtheorem{corollary}[theorem]{Corollary}
\newtheorem{claim}[theorem]{Claim}
\newtheorem{defn}[theorem]{Definition}
\newtheorem{ques}[theorem]{Question}
\newtheorem*{fact}{Facts}
\newtheorem{eg}[theorem]{Example}

\theoremstyle{plain}
\newtheorem{thmsub}{Theorem}[subsection]
\newtheorem{lemmasub}[thmsub]{Lemma}
\newtheorem{corollarysub}[thmsub]{Corollary}
\newtheorem{propositionsub}[thmsub]{Proposition}
\newtheorem{defnsub}[thmsub]{Definition}

\numberwithin{equation}{section}


\theoremstyle{remark}

\newtheorem{remark}[theorem]{Remark}
\newtheorem{remarks}{Remarks}
\newcommand*\widebar[1]{%
	\hbox{%
		\vbox{%
			\hrule height 0.5pt % The actual bar
			\kern0.6ex%         % Distance between bar and symbol
			\hbox{%
				\kern 0em%      % Shortening on the left side
				\ensuremath{#1}%
				\kern 0em%      % Shortening on the right side
			}%
		}%
	}%
}
%\renewcommand\thefootnote{\fnsymbol{footnote}}
%dont use number as footnote symbol, use this command to change

\DeclareMathOperator{\supp}{supp}
\DeclareMathOperator{\dist}{dist}
\DeclareMathOperator{\vol}{vol}
\DeclareMathOperator{\diag}{diag}
\DeclareMathOperator{\tr}{tr}
\DeclareMathOperator{\Img}{\operatorname{Im}}
\DeclareMathOperator{\Id}{\operatorname{Id}}
\DeclareMathOperator{\Rep}{\operatorname{Rep}}
\DeclareMathOperator{\Mod}{\operatorname{Mod}}
\DeclareMathOperator{\Hom}{\operatorname{Hom}}
\DeclareMathOperator{\Ext}{\operatorname{Ext}}
\DeclareMathOperator{\gldim}{\operatorname{gl.dim}}
\DeclareMathOperator{\projdim}{\operatorname{proj.dim}}
\DeclareMathOperator{\injdim}{\operatorname{inj.dim}}
\DeclareMathOperator{\dimv}{\operatorname{\underline{\mathbf{dim}}}}


\DeclareMathOperator{\Flagd}{\operatorname{Flag}_{\mathbf{d}}}
\DeclareMathOperator{\Flagdstr}{\operatorname{Flag}_{\mathbf{d},str}}
\newcommand{\Grr}{\operatorname{Gr}^{R}}
\newcommand{\Grq}{\operatorname{Gr}^{KQ}}
\newcommand{\Flag}[1]{\operatorname{Flag}_{\mathbf{#1}}}
\newcommand{\Flagstr}[1]{\operatorname{Flag}_{\mathbf{#1},str}}
\newcommand{\dimvec}[1]{\mathbf{#1}}

\begin{document}
\date{}

\title
{Affine paving of partial flag quiver variety}


\author{Xiaoxiang Zhou}
\address{School of Mathematical Sciences\\
University of Bonn\\
Bonn, 53115\\ Germany\\} 
\email{email:xx352229@mail.ustc.edu.cn}





\begin{abstract}
In this article, we establish an affine paving for partial flag quiver varieties when the quiver is of Dynkin type. By copying results in \cite[section 6]{irelli2019cell} word by word, the same problem for affine quiver reduced to the case where the representation is regular quasi-simple. The idea of the proof mainly comes from \cite{irelli2019cell}, and the result is a natural continuation of \cite{maksimau2019flag}.
\end{abstract}



\maketitle
\tableofcontents
%%%%%%%%%%%%%%%%%%%%%%%%%%%%%%%%%%%%%%%%%%%%%%%%%%%%%%%%%%%%%%%%%%%%%%%%%%%%%%%%%%%%%%%%%%%%%

\section{Introduction}
Let $Q$ be a quiver of Dynkin or affine type(without loops), $X\in \Rep(Q)$ be an quiver representation.\footnote{We fix the base field $K=\mathbb{C}$ for convinience.} We are interested in three objects related to $X \in \Rep(Q)$:
\begin{equation*}
\begin{aligned}
&	\text{quiver Grassmannian} && \Grq(X)\colon = \left\{ M_1 \mid M_1 \subseteq X \right\}\\
&	\text{partial flag variety \textcolor{gray}{$d\geqslant 1$}} && \Flagd(X)\colon =\left\{ 0 \subseteq M_1 \subseteq \cdots M_d \subseteq X \right\}\\
&	\text{strict partial flag variety \textcolor{gray}{$d\geqslant 2$}} && \Flagdstr(X)\colon =\left\{ 0 \subseteq M_1 \subseteq \cdots M_d \subseteq X  \mid x.M_{i+1} \subseteq M_i\right\}\footnotemark\\
\end{aligned}
\end{equation*}
\footnotetext{for any $x \in Q_1, i \in \{2,\ldots ,d \} $.}

It's easy to see that $\Flag{1}(X)=\Grq(X)$. These geometrical objects can be divided into different pieces according to the dimension vectors of $M_1,\ldots,M_d$, and each piece have its own natural (complex/Zarisky) topology. It was proved in \cite{irelli2019cell} that $\Grq(X)$ have an affine paving, and in \cite{maksimau2019flag} that $\Flagd(X)$ have the same property when $Q$ is Dynkin quiver of type $A/E$. Here we go one step further, the results are concluded in the ???.

???Here is one table

The idea of proof is very simple: first, we view the partial flag quiver variety as the quiver Grassmannian of the more complicated quiver; then we establish the decomposition so that one may solve the problem by induction; finally we set a special way of decomposition for each indecomposable module so that we can avoid meeting the bad decomposition. These contents are in Section ???, accordingly.

Before the end of this section, let us see in one example how partial flag variety is viewed as quiver Grassmannian.

\begin{eg}
Let $Q\colon x \longrightarrow y \longleftarrow z \longrightarrow w$ be a quiver, and let $X\colon X_x \longrightarrow X_y \longleftarrow X_z \longrightarrow X_w$ be a quiver representation, then $\Flag{3}(X),\Flagstr{3}$ can be viewed as quiver Grassmannian in ???:

???[pictures here]


\end{eg}
%%%%%%%%%%%%%%%%%%%%%%%%%%%%%%%%%%%%%%%%%%%%%%%%%%%%%%%%%%%%%%%%%%%%%%%%%%%%%%%%%%%%%%%%%%%%%

\section{Preliminary Facts}
Fix the quiver $Q$ ant the integer $d \geqslant 2$, we define the new bigger quiver $Q_{d},Q_{d,str}$ as follows:

\begin{itemize}
	\item The vertex set of $Q_d$(resp. $Q_{d,str}$) is defined as the Cartesian product of the vertex set of $Q$ and $\{1,\ldots,d\}$, i.e.
	$$v(Q_d)=v(Q_{d,str})=v(Q) \times \{1,\ldots,d\}$$
	\item  The arrows of $Q_d$(resp. $Q_{d,str}$) is defined as follows:
	\begin{itemize}
		\item for each $(i,r) \in v(Q) \times \{1,\ldots,d-1\}$, there is one arrow from $(i,r)$ to $(i,r+1)$;
		\item for each arrow $i \longrightarrow j$ in quiver $Q$:
		\begin{itemize}
			\item $Q_d$ case: there is one arrow from $(i,r)$ to $(j,r)$; \textcolor{gray}{(for any $r \in \{1,\ldots,d\}$)}
			\item $Q_{d,str}$ case: there is one arrow from $(i,r)$ to $(j,r-1)$; \textcolor{gray}{(for any $r \in \{1,\ldots,d-1\}$)}			
		\end{itemize}
	\end{itemize}
\end{itemize}
\begin{eg}
Here is the picture of new quiver:

???[pictures here]
\end{eg}

We define the ring 
$$R=??? or ???$$
and define the canonical map $\Phi:\Rep(Q) \longrightarrow \Mod(R) \textcolor{gray}{\subseteq \Rep(Q_d \text{ or } Q_{d,str})}$ as follows:
\begin{itemize}
\item $\left(\Phi(X)\right)_{(i,r)}:=X_i$;
\item $\left(\Phi(X)\right)_{(i,r) \rightarrow (i,r+1)}:=\Id_{X_i}$;
\item $\left(\Phi(X)\right)_{(i,r) \rightarrow (j,-)}:=X_{i \rightarrow j} \textcolor{gray}{ \quad(i \rightarrow j)}$;
\end{itemize}
For the module $T \in \Mod(R)$, we define $\Grr(T)\colon=\{T' \subseteq T \text{ as the submodule}  \}$.
\begin{proposition}
Fix the representation $X\in \Rep(Q)$, we have the isomorphism
$$\Flag{}(X)\cong \Grr(\Phi(X)).$$
\begin{proof}
The two side give us same amount of informations. Or, you can easily construct the bijection from $\Flag{}(X)$ to $\Grr(\Phi(X))$.
\end{proof}
\end{proposition}
\subsection{Ext-vanishing properties}
We would like to show some higher rank Extension group to be 0, which would be a key ingredient in the proof of the next section.

\begin{lemma}
Let $M,N,X,S \in \Rep(Q)$, $V,W,T \in \Mod(R)$.
\begin{enumerate}[(1)]
	\item $\gldim R \leqslant 2$;\label{lm:gldim}
	\item The functor $\Phi:\Rep(Q) \longrightarrow \Mod(R)$ is exact;
	\item $\Phi$ maps projective module to projective module;\label{lm:toproj}
	\item $\Hom_{KQ}(M,N) \cong \Hom_{R}(\Phi(M),\Phi(N)),\quad\Ext^i_{KQ}(M,N) \cong \Ext^i_{R}(\Phi(M),\Phi(N))$;
	\item $\projdim \Phi(M) \leqslant 1, \injdim \Phi(M) \leqslant 1$;
	\item Suppose $V \subseteq \Phi(X), W \subseteq \Phi(S),$ then $\Ext^2_{R}(W,T)=0, \Ext^2_{R}(T,\Phi(X)/V)=0$.
\end{enumerate}
\end{lemma}
\begin{proof}
For (\ref{lm:gldim}), we just need to check minimal projective resolution of $S(i)$ in $\Mod(R)$; for (\ref{lm:toproj}), we reduced to the case of indecomposable projective modules. The rests are easy exercises of homological algebra.
\end{proof}
\subsection{How much do we understand the quiver representation?}
To understand the category $\Rep(Q)$, one should understand indecomposable modules(as well as their relations). This has almost been done in the Auslander-Reiten theory. For example, when the quiver $Q$ is of Dynkin type, then there are only finite ind rep(up to isomorphism) and each ind rep corresponds to the positive root of Dynkin diagram. One can compute the Auslander-Reiten quiver by knitting algorithm and get the structure of ind rep. Moreover, one can directly get Hom space between $M$ and $N$ by looking at nontrivial paths from $M$ to $N$\footnote{These paths may be linear dependent, so it's not too easy.}.

We will use the Auslander-Reiten quiver to find "good monomorphisms" in Section ???. For more informations about Auslander-Reiten theory, one can see ???.
%%%%%%%%%%%%%%%%%%%%%%%%%%%%%%%%%%%%%%%%%%%%%%%%%%%%%%%%%%%%%%%%%%%%%%%%%%%%%%%%%%%%%%%%%%%%%

\section{Main Theorem}
   
   In the following sections, we always suppose that 
   \begin{itemize}
   		\item $X,Y,S,M,N \in \Rep(Q), V,U,W,T,T' \in \Mod(R)$;
   		\item we have the short exact sequence $\eta:0\longrightarrow X \longrightarrow Y \longrightarrow S \longrightarrow 0$, and $V \subseteq \Phi(X), U \subseteq \Phi(Y), W \subseteq \Phi(S)$;
   		\item $\dimvec{f},\dimvec{g}$ are dimension vectors of quiver $Q_d$ or $Q_{d,str}$. $\Grr_{\dimvec{f}}(T):= \{ T' \subseteq T \mid \dimv T'=\dimvec{f} \}$ is defined as the set of subrepresentations with dimension vector $\dimvec{f}$.
   \end{itemize}
We use 
\begin{equation*}
\begin{aligned}[]
	[T,T']^i:&=\dim_K \Ext^i_R (T,T'),\qquad [T,T']:=\dim_K \Hom_R (T,T')\\
	\left< T,T'\right>_R:&= \sum_{i=0}^{\infty} (-1)^i [T,T']^i \quad=[T,T']-[T,T']^1+[T,T']^2\\
	\left< \dimvec{f},\dimvec{g}\right>_R:&= \sum_{i \in v(R)} f_ig_i - \sum_{b \in a(R)} f_{s(b)}g_{t(b)}+ \sum_{c \in va(R)} f_{s(c)}g_{t(c)}
\end{aligned}
\end{equation*}
for the shorthand notation, where
\begin{equation*}
\begin{aligned}
	v(R):&= \{\text{vertices in quiver $KQ_d$ or $KQ_{d,str}$}\} \\
	a(R):&= \{\text{arrows in quiver $KQ_d$ or $KQ_{d,str}$}\} \\
	va(R):&= \{\text{"virtual arrows" in quiver $KQ_d$ or $KQ_{d,str}$}\} \\
\end{aligned}
\end{equation*}

???[pictures with caption "virtual arrow: can ne thought as the "face" of the quiver"]

Let $\eta: 0\longrightarrow X \stackrel{\iota}{\longrightarrow} Y \stackrel{\pi}{\longrightarrow} S \longrightarrow 0$ be a short exact sequence in $Rep(Q)$. Consider the canonical \textbf{non-continuous} map
$$\Psi: \Grr(\Phi(Y)) \longrightarrow \Grr(\Phi(X)) \times \Grr(\Phi(S)) \qquad U \longmapsto \left([\Phi(\iota)]^{-1}(U),[\Phi(\pi)](U)   \right)$$
and  $\Psi_{\dimvec{f},\dimvec{g}}$ is the map $\Psi$ restricted to the preimage of $\Grr_{\dimvec{f}}(\Phi(X)) \times \Grr_{\dimvec{g}}(\Phi(S))$.

The goal of this section is to prove the following theorems:
\begin{theorem}
	When $\eta$ splits, $\Psi$ is surjective. Moreover, $\Psi_{\dimvec{f},\dimvec{g}}$ is a Zarisky-locally trivial affine bundle of rank $\left< \dimvec{g},\dimv \Phi(X) - \dimvec{f}\right>_R$.
\end{theorem}
\begin{theorem}
	When $\eta$ does not split and $[S,X]^1=1$, 
	$$\Img \Psi_{\dimvec{f},\dimvec{g}} = \bigg(\Grr_{\dimvec{f}}(\Phi(X)) \times \Grr_{\dimvec{g}}(\Phi(S)) \bigg) \setminus \bigg(\Grr_{\dimvec{f}}(\Phi(X_S)) \times \Grr_{\dimvec{g}-\dimv \Phi(S^X)}\left(\Phi(S/S^X)\right) \bigg)$$
	Moreover, $\Psi_{\dimvec{f},\dimvec{g}}$ is a Zarisky-locally trivial affine bundle of rank $\left< \dimvec{g},\dimv \Phi(X) - \dimvec{f}\right>_R$ over $\Img \Psi_{\dimvec{f},\dimvec{g}}$.
\end{theorem}
%%%%%%%%%%%%%%%%%%%%%%%%%%%%%%%%%%%%%%%%%%%%%%%%%%%%%%%%%%%%%%%%%%%%%%%%%%

\section{Application: Dynkin Case}




%%%%%%%%%%%%%%%%%%%%%%%%%%%%%%%%%%%%%%%%%%%%%%%%%%%%%%%%%%%%%%%%%%%%%%%%%%%%%%%%%%%%%%%%%%%%%%%
\section{Application: Affine Case}



\bibliographystyle{plain}
\bibliography{reference}
\nocite{irelli2019cell}
\nocite{maksimau2019flag}




\end{document}




